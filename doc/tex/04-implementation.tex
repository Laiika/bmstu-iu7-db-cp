\chapter{Технологический раздел}

\section{Анализ систем управления базами данных}
Для базы данных обычно требуется комплексное программное обеспечение, которое называется системой управления базами данных (СУБД). СУБД служит интерфейсом между базой данных и пользователями или программами, предоставляя пользователям возможность получать и обновлять информацию, а также управлять ее упорядочением и оптимизацией. СУБД обеспечивает контроль и управление данными, позволяя выполнять различные административные операции, такие как мониторинг производительности, настройка, а также резервное копирование и восстановление~\cite{db}.

\subsection{PostgreSQL}
PostgreSQL --- реляционная СУБД с отрытым исходным кодом, предоставляет транзакции, обладающие свойствами ACID, автоматически обновляемые представления, материализованные представления, триггеры, внешние ключи и хранимые процедуры. Данная СУБД предназначена для обработки ряда рабочих нагрузок, от отдельных компьютеров до хранилищ данных или веб-сервисов с множеством одновременных пользователей. Лицензия PostgreSQL разрешает ее неограниченное использование, модификацию кода, а также включение в состав других продуктов, в том числе закрытых и коммерческих~\cite{postgresql}.

\subsection{MariaDB}
MariaDB --- реляционная СУБД, разработанная под руководством разработчиков оригинальной версии MySQL. MariaDB является преемницей СУБД MySQL и практически полностью совместима с ней. API и протоколы в MariaDB соответствуют используемым в MySQL, поэтому все библиотеки и приложения, которые работают в MySQL, должны работать и с MariaDB. Лицензируется по свободной лицензии GNU GPL~\cite{mariadb}.

\subsection{ClickHouse}
ClickHouse --- столбцовая СУБД для онлайн обработки аналитических запросов (OLAP) с открытым кодом, разрабатываемая компанией Яндекс. ClickHouse поддерживает декларативный язык запросов на основе SQL и во многих случаях совпадающий с SQL стандартом. При том, что ClickHouse является реляционной СУБД, он не поддерживает транзакции, а также точечные операции UPDATE и DELETE. Поддерживаются GROUP BY, ORDER BY, подзапросы в секциях FROM, IN, JOIN, функции window, а также скалярные подзапросы~\cite{clickhouse}.

\subsection{Выбор СУБД для решения задачи}
Для решения поставленной задачи была выбрана СУБД PostgreSQL, потому что данная СУБД является бесплатной, поддерживает ACID транзакции, позволяет создавать  функции и проста в развертывании.

%Можно выделить основные требования, которым должна удовлетворять СУБД для решения поставленной задачи:
%\begin{itemize}
	%\item соответствие требованиям ACID;
	%\item бесплатное использование;
	%\item возможность создания функций.
%\end{itemize}

%Результаты сравнения СУБД приведены в таблице \ref{tab:dsm}.

%\begin{table}[h]
%    \caption{Сравнение СУБД}
 %   \begin{center}
  %      \begin{tabular}{|l|l|l|l|}
   %         \hline
    %        \textbf{СУБД} & \textbf{Т1} & \textbf{Т2} & \textbf{Т3} \\ \hline
     %       PostgreSQL & + & + & + \\ \hline
      %      MariaDB & + & + & + \\ \hline
       %     ClickHouse & - & + & + \\ \hline
        %\end{tabular}
   % \end{center}
   % \label{tab:dsm}
%\end{table}

%По результатам сравнения можно сделать вывод о том, что PostgreSQL удовлетворяет всем перечисленным требованиям.


\section{Средства реализации}


    
\section*{Вывод}

В данном разделе был проведен анализ СУБД, в результате которого была выбрана СУБД PostgreSQL.