\chapter{Аналитический раздел}

\section{Формализация задачи}
Необходимо спроектировать и реализовать базу данных для хранения и обработки показаний приборов газового анализа. Также необходимо разработать интерфейс, позволяющий работать с данной базой для получения и изменения хранящейся в ней информации и формирования отчетов с выборками данных по заданным параметрам. Требуется реализовать, как минимум, три вида ролей --- пользователь, сотрудник и администратор.


\section{Формализация данных}
База данных должна хранить информацию о следующих данных:

\begin{itemize}
	\item датчики;
	\item события;
	\item пользователи.
\end{itemize}

В таблице \ref{tab:types} приведены группы данных и сведения о них.

\begin{table}[h]
    \caption{Группы данных и сведения о них}
    \begin{center}
        \begin{tabular}{|l|l|}
            \hline
            \textbf{Группа} & \textbf{Сведения} \\ \hline
            Датчики & \begin{tabular}[c]{@{}l@{}}ID датчика (серийный номер), тип, калибровочный газ,\\ калибровка концентрации газа, сигнализация нижнего\\ предела, сигнализация верхнего предела\end{tabular} \\ \hline
            События & \begin{tabular}[c]{@{}l@{}}ID события, время сигнала, газ, серийный номер датчика,\\ сигнализация нижнего предела, сигнализация верхнего\\ предела\end{tabular} \\ \hline
            Пользователи & ID пользователя, email (логин), пароль \\ \hline
        \end{tabular}
    \end{center}
    \label{tab:types}
\end{table}

На рисунке \ref{img:ER} представлена ER-диаграмма сущностей проектируемой базы данных в нотации Чена, основанная на приведенной выше таблице.

\includeimage
    {ER}
    {f}
    {h}
    {0.8\textwidth}
    {ER-диаграмма}


\section{Формализация ролей}
В соответствии с поставленной задачей необходимо разработать приложение с возможностью аутентификации пользователей, что делит их на авторизованных и неавторизованных. Для управления приложением необходима ролевая модель: авторизованный пользователь, сотрудник и администратор. Для каждой роли предусмотрен свой набор функций.

Набор функций неавторизованного пользователя:

\begin{itemize}
	\item аутентификация;
	\item регистрация аккаунта пользователя.
\end{itemize}

Диаграмма вариантов использования для неавторизованного пользователя представлена на рисунке \ref{img:use-case-unautho}.
\clearpage

\includeimage
    {use-case-unautho}
    {f}
    {h}
    {0.8\textwidth}
    {Use-case диаграмма для неавторизованного пользователя}

Набор функций авторизованного пользователя:

\begin{itemize}
	\item выход из аккаунта;
        \item получение данных о событиях на датчиках по заданным параметрам:
        \begin{itemize}
            \item газ;
            \item время сигнала;
            \item тип события (?).
        \end{itemize}
\end{itemize}

Диаграмма вариантов использования для авторизованного пользователя представлена на рисунке \ref{img:use-case-autho}.

\includeimage
    {use-case-autho}
    {f}
    {h}
    {0.8\textwidth}
    {Use-case диаграмма для авторизованного пользователя}

Набор функций сотрудника:

\begin{itemize}
	\item выход из аккаунта;
        \item получение данных о событиях на датчиках по заданным параметрам:
        \begin{itemize}
            \item газ;
            \item время сигнала;
            \item тип события (?);
        \end{itemize}
        \item добавление и удаление событий;
        \item добавление и удаление датчиков;
        \item создание отчета с заданной выборкой данных из базы.
\end{itemize}

Диаграмма вариантов использования для сотрудника представлена на рисунке \ref{img:use-case-empl}.

\includeimage
    {use-case-empl}
    {f}
    {h}
    {0.8\textwidth}
    {Use-case диаграмма для сотрудника}

Набор функций администратора:

\begin{itemize}
	\item выход из аккаунта;
        \item регистрация сотрудника;
        \item регистрация администратора;
        \item удаление аккаунта пользователя или сотрудника.
\end{itemize}

Диаграмма вариантов использования для администратора представлена на рисунке \ref{img:use-case-admin}.

\includeimage
    {use-case-admin}
    {f}
    {h}
    {0.8\textwidth}
    {Use-case диаграмма для администратора}
    

\section{Анализ баз данных}

База данных --- это упорядоченный набор структурированной информации или данных, которые обычно хранятся в электронном виде в компьютерной системе~\cite{db}.

Существует три группы моделей баз данных, отличающиеся структурой организации данных --- дореляционные, реляционные и постреляционные базы данных.

\subsection{Дореляционные базы данных}
Ранние модели баз данных называются дореляционными~\cite{before}. К ним относятся:

\begin{itemize}
	\item иерархическая --- система, в основе которой лежит древовидная структура;
	\item сетевая --- модель данных, представляющая взаимосвязи элементов данных в виде ориентированного графа;
	\item модель, в основе которой лежат инвертированные списки.
\end{itemize}

Преимущества дореляционных баз данных:

\begin{itemize}
	\item возможность экономии памяти компьютера;
	\item возможность построения вручную эффективных прикладных систем.
\end{itemize}

Недостатки дореляционных баз данных:

\begin{itemize}
	\item сложность использования;
        \item необходимы знания о физической организации;
	\item прикладные системы зависят от этой организации.
\end{itemize}

\subsection{Реляционные базы данных}
 Реляционные базы данных основаны на реляционной модели — интуитивно понятном, наглядном табличном способе представления данных. Каждая строка, содержащая в таблице такой базы данных, представляет собой запись с уникальным идентификатором, который называют ключом. Столбцы таблицы имеют атрибуты данных, а каждая запись обычно содержит значение для каждого атрибута, что дает возможность легко устанавливать взаимосвязь между элементами данных~\cite{rel}.

Для связывания информации из разных таблиц используются внешние ключи --- уникальные идентификаторы атомарного фрагмента данных в этой таблице. Другие таблицы могут ссылаться на этот внешний ключ, чтобы создать связь между частями данных и частью, на которую указывает внешний ключ.

Реляционные базы данных обеспечивают набор свойств ACID:

\begin{itemize}
	\item атомарность --- каждая транзакция будет выполнена полностью или не будет выполнена совсем;
	\item непротиворечивость --- по завершении транзакции данные должны соответствовать схеме базы данных;
	\item изолированность --- параллельные транзакции не будут оказывать влияния на результат других транзакций;
	\item надежность --- изменения, получившиеся в результате транзакции, должны оставаться сохраненными вне зависимости от каких-либо сбоев.
\end{itemize}

Преимущества реляционных баз данных:

\begin{itemize}
	\item наглядный способ представления данных;
	\item эффективная поддержка целостности данных.
\end{itemize}

Недостатки реляционных баз данных:

\begin{itemize}
	\item низкая скорость доступа к данным;
	\item сложность описания иерархических и сетевых связей.
\end{itemize}

\subsection{Постреляционные базы данных}

Для постреляционных моделей данных появляется возможность многозначных полей \cite{after}. Набор подзначений становится самостоятельной таблицей, встроенной в главную. Информация может быть представлена как:

\begin{itemize}
	\item пары <<ключ-значение>>;
	\item документы, упорядоченные по группам, называемым коллекциями;
	\item граф.
\end{itemize}

Преимущества постреляционных баз данных:

\begin{itemize}
	\item возможность формирования совокупности связанных реляционных таблиц через одну постреляционную таблицу, что обеспечивает эффективность ее обработки;
	\item отсутствие ограничений на типы данных.
\end{itemize}

Недостатки постреляционных баз данных:

\begin{itemize}
	\item несовместимость с запросами SQL;
	\item сложность решения проблемы обеспечения целостности и непротиворечивости хранимых данных.
\end{itemize}

\subsection{Выбор модели данных}

В соответствии с формализацией задачи можно выделить следующие требования к разрабатываемой базе данных:

\begin{itemize}
	\item поддержка принципов ACID;
	\item совместимость с запросами SQL.
\end{itemize}

Исходя из требований к разрабатываемой базе данных можно сделать вывод о том, что для решения задачи предпочтительной является реляционная база данных.

\section*{Вывод}
В данном разделе были определены группы данных, выделены роли для управления приложением и описаны функции каждой из ролей. Было произведено сравнение моделей баз данных, для реализации была выбрана реляционная модель данных.
